%! Author = Raffaele
%! Date = 11/01/2024

\thispagestyle{headings}
\section{Sintesi dei requisiti}\label{sec:}

\subsection{Requisiti funzionali}\label{subsec:requisiti-funzionali}
\begin{itemize}
    \item Il programma deve poter simulare una partita tra l'utente e un avversario finto le cui mosse
    sono effettuate dall'applicazione stessa sfruttando un particolare algoritmo selezionato opportunamente.
    Per semplicità, ci si riferirà a tale avversario finto semplicemente come \("\)Computer\("\).
    \item Le pedine sono rappresentate nella griglia da un colore, il rosso simboleggia una pedina controllata
    dall'utente, il giallo simboleggia una pedina controllata dal computer.
    \item L'utente deve poter inserire all'interno della griglia di gioco una propria pedina in una qualsiasi colonna
    libera, il programma deve rispondere inserendo una pedina controllata dal computer.
    \item Seguendo le regole del gioco classico, nel momento in cui quattro pedine dello stesso colore sono in fila
    in una qualsiasi direzione, il gioco termina e il controllore di quelle pedine vince la partita. \\
    Se la partita è stata vinta dall'utente, il risultato deve essere salvato.
    \item L'utente deve essere in grado di autenticarsi all'interno \\dell'applicazione inserendo nome e cognome. \\
    Dopo aver inserito tali dati, il programma deve registrarlo opportunamente, in modo che i
    risultati delle partite vengano memorizzati.
    \item L'utente deve poter scegliere una modalità di gioco, che simboleggia il modo in cui il computer inserisce
    le proprie pedine.
    Le modalità disponibili sono le seguenti:
        \begin{itemize}
            \item \textbf{Difesa:} Il computer controlla il numero massimo di pedine in fila del giocatore, e inserisce
            una propria pedina per tentare di bloccare la sequenza.
            \item \textbf{Attacco:} Il computer utilizza l'algoritmo \textbf{Minimax con Alpha-beta pruning}\cite{Wiki} per dare
            una valutazione alle mosse attualmente disponibili - dove per mossa si intende l'inserimento di una propria
            pedina all'interno di una colonna - e giocare la mossa con la valutazione più alta. \\
            L'algoritmo in questione verrà discusso nel dettaglio in una sezione successiva.
            \item \textbf{Neutrale:} Il computer effettua una mossa scegliendo in maniera casuale tra la modalità Difesa
            e Attacco.
        \end{itemize}
    \item Il risultato di una partita deve essere memorizzato dal programma, deve essere inoltre possibile per l'utente
    visualizzare una classifica degli utenti che hanno totalizzato i punteggi più altri, dove per punteggio si intende
    il numero di vittorie.
    \item Deve essere possibile per l'utente mettere in pausa una partita in un qualsiasi momento per riprenderla
    successivamente.
\end{itemize}

\subsection{Requisiti non funzionali}\label{subsec:requisiti-non-funzionali}
\begin{itemize}
    \item L'applicazione deve essere sviluppata utilizzando il linguaggio \textbf{Java} e attenendosi ai principi
    della programmazione \textbf{SOLID}.
    \item L'applicazione deve prevedere un'interfaccia grafica per rendere agevole l'interazione con l'utente.
    \item E' necessario utilizzare almeno due Design Pattern tra quelli studiati.
    \item I dati devono essere memorizzati tramite file o database.
\end{itemize}