%! Author = Raffaele
%! Date = 12/01/2024

\thispagestyle{headings}
\newpage
\section{Conclusioni}\label{sec:conclusioni}

In questo progetto è stato sviluppato un programma in grado di simulare il gioco da tavolo \textit{Forza quattro}. \\
L'utilizzo del Design Pattern \textbf{Strategy} ha permesso di disaccoppiare la logica centrale del funzionamento
di una partita dalla modalità di gioco che definisce il comportamento del computer. \\
Grazie a questo pattern in futuro sarà possibile aggiungere nuove modalità di gioco in maniera estremamente facile,
semplicemente implementando l'interfaccia \textit{ComputerStrategy}, e senza dover modificare le modalità esistenti
in alcun modo.
\\
L'utilizzo del Design Pattern \textbf{Template Method} ha permesso di definire un algoritmo per standaridzzare le pagine
dell'interfaccia grafica, grazie a questo pattern il programma può facilmente supportare nuove pagine della GUI senza
dover modificare le pagine esistenti, inoltre se si dovesse decidere di cambiare il look and feel, basterebbe modificare
soltanto il metodo all'interno della classe \textit{PageTemplate} piuttosto che in ogni singola pagina.
\\
L'utilizzo del Design Pattern \textbf{Singleton} ha permesso di definire un'unica istanza delle classi GameHandler e
Controller, fornendo un punto di accesso globale ad esse. \\
Grazie a questo pattern diventa impossibile per altre classi istanziare inavvertitamente un secondo GameHandler o
Controller, cosa che \\porterebbe soltanto ad errori. \\